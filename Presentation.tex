% Options for packages loaded elsewhere
\PassOptionsToPackage{unicode}{hyperref}
\PassOptionsToPackage{hyphens}{url}
%
\documentclass[
  ignorenonframetext,
  t]{beamer}
\usepackage{pgfpages}
\setbeamertemplate{caption}[numbered]
\setbeamertemplate{caption label separator}{: }
\setbeamercolor{caption name}{fg=normal text.fg}
\beamertemplatenavigationsymbolsempty
% Prevent slide breaks in the middle of a paragraph
\widowpenalties 1 10000
\raggedbottom
\setbeamertemplate{part page}{
  \centering
  \begin{beamercolorbox}[sep=16pt,center]{part title}
    \usebeamerfont{part title}\insertpart\par
  \end{beamercolorbox}
}
\setbeamertemplate{section page}{
  \centering
  \begin{beamercolorbox}[sep=12pt,center]{section title}
    \usebeamerfont{section title}\insertsection\par
  \end{beamercolorbox}
}
\setbeamertemplate{subsection page}{
  \centering
  \begin{beamercolorbox}[sep=8pt,center]{subsection title}
    \usebeamerfont{subsection title}\insertsubsection\par
  \end{beamercolorbox}
}
\AtBeginPart{
  \frame{\partpage}
}
\AtBeginSection{
  \ifbibliography
  \else
    \frame{\sectionpage}
  \fi
}
\AtBeginSubsection{
  \frame{\subsectionpage}
}
\usepackage{amsmath,amssymb}
\usepackage{iftex}
\ifPDFTeX
  \usepackage[T1]{fontenc}
  \usepackage[utf8]{inputenc}
  \usepackage{textcomp} % provide euro and other symbols
\else % if luatex or xetex
  \usepackage{unicode-math} % this also loads fontspec
  \defaultfontfeatures{Scale=MatchLowercase}
  \defaultfontfeatures[\rmfamily]{Ligatures=TeX,Scale=1}
\fi
\usepackage{lmodern}
\ifPDFTeX\else
  % xetex/luatex font selection
\fi
% Use upquote if available, for straight quotes in verbatim environments
\IfFileExists{upquote.sty}{\usepackage{upquote}}{}
\IfFileExists{microtype.sty}{% use microtype if available
  \usepackage[]{microtype}
  \UseMicrotypeSet[protrusion]{basicmath} % disable protrusion for tt fonts
}{}
\makeatletter
\@ifundefined{KOMAClassName}{% if non-KOMA class
  \IfFileExists{parskip.sty}{%
    \usepackage{parskip}
  }{% else
    \setlength{\parindent}{0pt}
    \setlength{\parskip}{6pt plus 2pt minus 1pt}}
}{% if KOMA class
  \KOMAoptions{parskip=half}}
\makeatother
\usepackage{xcolor}
\newif\ifbibliography
\usepackage{graphicx}
\makeatletter
\newsavebox\pandoc@box
\newcommand*\pandocbounded[1]{% scales image to fit in text height/width
  \sbox\pandoc@box{#1}%
  \Gscale@div\@tempa{\textheight}{\dimexpr\ht\pandoc@box+\dp\pandoc@box\relax}%
  \Gscale@div\@tempb{\linewidth}{\wd\pandoc@box}%
  \ifdim\@tempb\p@<\@tempa\p@\let\@tempa\@tempb\fi% select the smaller of both
  \ifdim\@tempa\p@<\p@\scalebox{\@tempa}{\usebox\pandoc@box}%
  \else\usebox{\pandoc@box}%
  \fi%
}
% Set default figure placement to htbp
\def\fps@figure{htbp}
\makeatother
\setlength{\emergencystretch}{3em} % prevent overfull lines
\providecommand{\tightlist}{%
  \setlength{\itemsep}{0pt}\setlength{\parskip}{0pt}}
\setcounter{secnumdepth}{-\maxdimen} % remove section numbering
\usepackage{bookmark}
\IfFileExists{xurl.sty}{\usepackage{xurl}}{} % add URL line breaks if available
\urlstyle{same}
\hypersetup{
  pdftitle={Group 3: Predicting WAR},
  pdfauthor={Gioia Bonanno-Garcia, Ari Crumley, Vincent West},
  hidelinks,
  pdfcreator={LaTeX via pandoc}}

\title{Group 3: Predicting WAR}
\author{Gioia Bonanno-Garcia, Ari Crumley, Vincent West}
\date{2025-12-03}

\begin{document}
\frame{\titlepage}

\begin{frame}{Overview}
\phantomsection\label{overview}
\vspace{0.5cm}

\begin{columns}[T]
\begin{column}{0.48\linewidth}
\begin{itemize}
\tightlist
\item
  Goal: build a model to predict WAR (Wins Above Replacement)
\item
  \textbf{WAR measures a player’s total contribution} compared to a
  replacement-level player

  \begin{itemize}
  \tightlist
  \item
    It combines multiple aspects of performance: hitting, baserunning,
    defensive value, positional difficulty, and their playing time, and
    puts it into a single number.
  \item
    Interpreted as the number of additional wins a player adds to a team
  \item
    Since it provides a single encompassing measure of a player's value,
    WAR is greatly relied on by MLB front offices
  \end{itemize}
\end{itemize}
\end{column}

\begin{column}{0.48\linewidth}
\pandocbounded{\includegraphics[keepaspectratio]{Pictures/war-rating-chart.jpg}}
\end{column}
\end{columns}
\end{frame}

\begin{frame}{Motivation}
\phantomsection\label{motivation}
\vspace{0.5cm}

\begin{itemize}
\item
  Can a player's current-season performance statistics be used to
  predict their next season Wins Above Replacement (WAR)?
\item
  Major League Baseball teams rely heavily on WAR to evaluate a players
  value, make contract decisions, and project roster needs.
\item
  It provide a competitive advantage for a team by:

  \begin{itemize}
  \tightlist
  \item
    Helps identify declining players
  \item
    Allows for budgeting and contract planning
  \item
    Encourages player development and roster optimization
  \end{itemize}
\item
  Our goal is, by using statistical and machine learning models, we
  will:

  \begin{itemize}
  \tightlist
  \item
    Identify which player statistics best predict future WAR
  \item
    Compare the performance of three different statistical learning
    models: OLS, LASSO, and BOOSTING
  \item
    Build a model that maintains high predictive accuracy
  \end{itemize}
\end{itemize}
\end{frame}

\begin{frame}{Data used}
\phantomsection\label{data-used}
\vspace{0.7em}

\begin{itemize}
\item
  Our data came from baseball-reference.com
\item
  We used standard batting data from 2020-2025 to build our models (500
  obs each year)

  \begin{itemize}
  \tightlist
  \item
    2020-2024 was used for training
  \item
    2025 was our test data set
  \end{itemize}
\item
  Variables:

  \begin{itemize}
  \tightlist
  \item
    WAR, age, games played, plate appearances, at bats, runs scored,
    hits, doubles, triples, home runs, RBIs, stolen bases, caught
    stealing, walks, strikeouts, batting average, on base percentage,
    slugging percentage, OPS percentage, OPS+, rOBA, Rbat+, total bases,
    double plays grounded into, hit by pitch, sacrifice hits, sacrifice
    flies, intentional walks
  \end{itemize}
\end{itemize}
\end{frame}

\begin{frame}{Models Used}
\phantomsection\label{models-used}
\vspace{0.5cm}

\begin{columns}[T]
\begin{column}{0.333\linewidth}
\begin{itemize}
\tightlist
\item
  OLS

  \begin{itemize}
  \tightlist
  \item
    Baseline linear model
  \item
    Benchmark for comparing models
  \end{itemize}
\end{itemize}
\end{column}

\begin{column}{0.333\linewidth}
\begin{itemize}
\tightlist
\item
  LASSO

  \begin{itemize}
  \tightlist
  \item
    Feature selection through L1 regularization
  \end{itemize}
\end{itemize}
\end{column}

\begin{column}{0.333\linewidth}
\begin{itemize}
\tightlist
\item
  Boosting

  \begin{itemize}
  \tightlist
  \item
    Tree based method
  \item
    Good at capturing non-linear patterns
  \end{itemize}
\end{itemize}
\end{column}
\end{columns}
\end{frame}

\begin{frame}{OLS Models}
\phantomsection\label{ols-models}
\vspace{0.5cm}

\begin{itemize}
\item
  OLS identifies and measures the relationship between a response
  variable and predictor variables.

  \begin{itemize}
  \tightlist
  \item
    Finds a best-fitting line through a set of data points
  \end{itemize}
\item
  Pros: Convenient, accurate regression results for linearly related
  data
\item
  Cons: May be too simplistic for real world examples, assumptions of
  Linear Regression
\end{itemize}
\end{frame}

\begin{frame}{OLS Metrics Plot}
\phantomsection\label{ols-metrics-plot}
\vspace{0.5cm}

\begin{columns}[T]
\begin{column}{0.4\linewidth}
\begin{itemize}
\tightlist
\item
  Error metrics for both final testing data set (2025) and the training
  data split into training and testing sets
\item
  Significant terms:

  \begin{itemize}
  \tightlist
  \item
    age, games played, plate appearances, at bats, runs, hits, doubles,
    triples, home runs, stolen bases, caught stealing, walks,
    strikeouts, OPS+
  \end{itemize}
\end{itemize}
\end{column}

\begin{column}{0.6\linewidth}
\begin{center}\includegraphics[width=0.9\linewidth]{Pictures/OLS_comp_plot} \end{center}

\begin{frame}{LASSO Models}
\phantomsection\label{lasso-models}
\vspace{0.5cm}

\begin{itemize}
\item
  LASSO models perform regularization (L1), which shrinks some
  coefficients to exactly zero

  \begin{itemize}
  \tightlist
  \item
    Essentially feature selection
  \end{itemize}
\item
  Pros: Produces a more interpretative model, prevents over fitting
\item
  Cons: LASSO performs poorly when predictors are highly correlated
\end{itemize}
\end{frame}

\begin{frame}{LASSO Metrics Plot}
\phantomsection\label{lasso-metrics-plot}
\vspace{0.5cm}

\begin{columns}[T]
\begin{column}{0.4\linewidth}
\begin{itemize}
\tightlist
\item
  Error metrics for both final testing data set (2025) and the training
  data split into training and testing sets
\item
  Shrunk terms:

  \begin{itemize}
  \tightlist
  \item
    Plate appearances, home runs, RBIs, batting average, on base
    percentage, OPS+, rOBA
  \end{itemize}
\end{itemize}
\end{column}

\begin{column}{0.6\linewidth}
\begin{center}\includegraphics[width=0.9\linewidth]{Pictures/LASSO_comp_plot} \end{center}
\end{column}
\end{columns}
\end{frame}

\begin{frame}{Boosting Models}
\phantomsection\label{boosting-models}
\vspace{0.5cm}

\begin{itemize}
\item
  Boosting grows trees sequentially using information from previously
  grown trees

  \begin{itemize}
  \tightlist
  \item
    Each tree fit on a modified version of the original data set
  \end{itemize}
\end{itemize}

\vspace{0.4cm}

\begin{itemize}
\item
  Pros: High predictive accuracy and captures complex, nonlinear
  relationships automatically.
\item
  Cons: Prone to overfitting and requires careful tuning of
  hyperparameters to perform well.
\end{itemize}
\end{frame}

\begin{frame}{Boosting Metrics Plot}
\phantomsection\label{boosting-metrics-plot}
\vspace{0.5cm}

\begin{columns}[T]
\begin{column}{0.4\linewidth}
\begin{itemize}
\tightlist
\item
  Error metrics for both final testing data set (2025) and the training
  data split into training and testing sets
\end{itemize}
\end{column}

\begin{column}{0.6\linewidth}
\begin{center}\includegraphics[width=0.9\linewidth]{Pictures/boost_comp_plot} \end{center}
\end{column}
\end{columns}
\end{frame}

\begin{frame}{Model Comparison: OLS, LASSO, and Boosting}
\phantomsection\label{model-comparison-ols-lasso-and-boosting}
\begin{center}\includegraphics[width=0.33\linewidth]{Pictures/OLS Model Actual vs Predicted} \end{center}

\begin{center}\includegraphics[width=0.33\linewidth]{Pictures/Lasso Actual vs Predicted 2025} \end{center}

\begin{center}\includegraphics[width=0.33\linewidth]{Pictures/Boosting Actual vs Predicted 2025} \end{center}
\end{frame}
\end{column}
\end{columns}
\end{frame}

\begin{frame}{Model Comparison: RMSE, MAE, and R²}
\phantomsection\label{model-comparison-rmse-mae-and-ruxb2}
\centering

\includegraphics[height=0.23\textheight]{Pictures/Model_Comparison_RMSE_cropped.png}

\vspace{0.7em}

\includegraphics[height=0.23\textheight]{Pictures/Model_Comparison_MAE_cropped.png}

\vspace{0.7em}

\includegraphics[height=0.23\textheight]{Pictures/Model_Comparison_R2_cropped.png}
\end{frame}

\begin{frame}{Player Examples Using OLS model}
\phantomsection\label{player-examples-using-ols-model}
\vspace{0.7em}

\pandocbounded{\includegraphics[keepaspectratio]{Pictures/examples.png}}
\#\# Key Findings \vspace{0.5cm}

\begin{itemize}
\tightlist
\item
  OLS performed the best overall

  \begin{itemize}
  \tightlist
  \item
    Lowest prediction error
  \item
    Highest explained variance
  \end{itemize}
\item
  LASSO selected a subset of meaningful predictors, making it easier to
  understand which player stats drive WAR
\item
  Variables with a strong predictive value: plate apperances, home runs,
  hits, OPS+, walks, and strikouts
\item
  WAR prediction is challenging

  \begin{itemize}
  \tightlist
  \item
    some components are hard to obtain from batting-only statistics
  \item
    player injuries, playing time, or other extrnal factors produce
    noise
  \end{itemize}
\end{itemize}
\end{frame}

\begin{frame}{Questions?}
\phantomsection\label{questions}
\end{frame}

\end{document}
