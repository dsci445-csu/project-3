% Options for packages loaded elsewhere
\PassOptionsToPackage{unicode}{hyperref}
\PassOptionsToPackage{hyphens}{url}
%
\documentclass[
  ignorenonframetext,
  t]{beamer}
\usepackage{pgfpages}
\setbeamertemplate{caption}[numbered]
\setbeamertemplate{caption label separator}{: }
\setbeamercolor{caption name}{fg=normal text.fg}
\beamertemplatenavigationsymbolsempty
% Prevent slide breaks in the middle of a paragraph
\widowpenalties 1 10000
\raggedbottom
\setbeamertemplate{part page}{
  \centering
  \begin{beamercolorbox}[sep=16pt,center]{part title}
    \usebeamerfont{part title}\insertpart\par
  \end{beamercolorbox}
}
\setbeamertemplate{section page}{
  \centering
  \begin{beamercolorbox}[sep=12pt,center]{section title}
    \usebeamerfont{section title}\insertsection\par
  \end{beamercolorbox}
}
\setbeamertemplate{subsection page}{
  \centering
  \begin{beamercolorbox}[sep=8pt,center]{subsection title}
    \usebeamerfont{subsection title}\insertsubsection\par
  \end{beamercolorbox}
}
\AtBeginPart{
  \frame{\partpage}
}
\AtBeginSection{
  \ifbibliography
  \else
    \frame{\sectionpage}
  \fi
}
\AtBeginSubsection{
  \frame{\subsectionpage}
}
\usepackage{amsmath,amssymb}
\usepackage{iftex}
\ifPDFTeX
  \usepackage[T1]{fontenc}
  \usepackage[utf8]{inputenc}
  \usepackage{textcomp} % provide euro and other symbols
\else % if luatex or xetex
  \usepackage{unicode-math} % this also loads fontspec
  \defaultfontfeatures{Scale=MatchLowercase}
  \defaultfontfeatures[\rmfamily]{Ligatures=TeX,Scale=1}
\fi
\usepackage{lmodern}
\ifPDFTeX\else
  % xetex/luatex font selection
\fi
% Use upquote if available, for straight quotes in verbatim environments
\IfFileExists{upquote.sty}{\usepackage{upquote}}{}
\IfFileExists{microtype.sty}{% use microtype if available
  \usepackage[]{microtype}
  \UseMicrotypeSet[protrusion]{basicmath} % disable protrusion for tt fonts
}{}
\makeatletter
\@ifundefined{KOMAClassName}{% if non-KOMA class
  \IfFileExists{parskip.sty}{%
    \usepackage{parskip}
  }{% else
    \setlength{\parindent}{0pt}
    \setlength{\parskip}{6pt plus 2pt minus 1pt}}
}{% if KOMA class
  \KOMAoptions{parskip=half}}
\makeatother
\usepackage{xcolor}
\newif\ifbibliography
\usepackage{color}
\usepackage{fancyvrb}
\newcommand{\VerbBar}{|}
\newcommand{\VERB}{\Verb[commandchars=\\\{\}]}
\DefineVerbatimEnvironment{Highlighting}{Verbatim}{commandchars=\\\{\}}
% Add ',fontsize=\small' for more characters per line
\usepackage{framed}
\definecolor{shadecolor}{RGB}{248,248,248}
\newenvironment{Shaded}{\begin{snugshade}}{\end{snugshade}}
\newcommand{\AlertTok}[1]{\textcolor[rgb]{0.94,0.16,0.16}{#1}}
\newcommand{\AnnotationTok}[1]{\textcolor[rgb]{0.56,0.35,0.01}{\textbf{\textit{#1}}}}
\newcommand{\AttributeTok}[1]{\textcolor[rgb]{0.13,0.29,0.53}{#1}}
\newcommand{\BaseNTok}[1]{\textcolor[rgb]{0.00,0.00,0.81}{#1}}
\newcommand{\BuiltInTok}[1]{#1}
\newcommand{\CharTok}[1]{\textcolor[rgb]{0.31,0.60,0.02}{#1}}
\newcommand{\CommentTok}[1]{\textcolor[rgb]{0.56,0.35,0.01}{\textit{#1}}}
\newcommand{\CommentVarTok}[1]{\textcolor[rgb]{0.56,0.35,0.01}{\textbf{\textit{#1}}}}
\newcommand{\ConstantTok}[1]{\textcolor[rgb]{0.56,0.35,0.01}{#1}}
\newcommand{\ControlFlowTok}[1]{\textcolor[rgb]{0.13,0.29,0.53}{\textbf{#1}}}
\newcommand{\DataTypeTok}[1]{\textcolor[rgb]{0.13,0.29,0.53}{#1}}
\newcommand{\DecValTok}[1]{\textcolor[rgb]{0.00,0.00,0.81}{#1}}
\newcommand{\DocumentationTok}[1]{\textcolor[rgb]{0.56,0.35,0.01}{\textbf{\textit{#1}}}}
\newcommand{\ErrorTok}[1]{\textcolor[rgb]{0.64,0.00,0.00}{\textbf{#1}}}
\newcommand{\ExtensionTok}[1]{#1}
\newcommand{\FloatTok}[1]{\textcolor[rgb]{0.00,0.00,0.81}{#1}}
\newcommand{\FunctionTok}[1]{\textcolor[rgb]{0.13,0.29,0.53}{\textbf{#1}}}
\newcommand{\ImportTok}[1]{#1}
\newcommand{\InformationTok}[1]{\textcolor[rgb]{0.56,0.35,0.01}{\textbf{\textit{#1}}}}
\newcommand{\KeywordTok}[1]{\textcolor[rgb]{0.13,0.29,0.53}{\textbf{#1}}}
\newcommand{\NormalTok}[1]{#1}
\newcommand{\OperatorTok}[1]{\textcolor[rgb]{0.81,0.36,0.00}{\textbf{#1}}}
\newcommand{\OtherTok}[1]{\textcolor[rgb]{0.56,0.35,0.01}{#1}}
\newcommand{\PreprocessorTok}[1]{\textcolor[rgb]{0.56,0.35,0.01}{\textit{#1}}}
\newcommand{\RegionMarkerTok}[1]{#1}
\newcommand{\SpecialCharTok}[1]{\textcolor[rgb]{0.81,0.36,0.00}{\textbf{#1}}}
\newcommand{\SpecialStringTok}[1]{\textcolor[rgb]{0.31,0.60,0.02}{#1}}
\newcommand{\StringTok}[1]{\textcolor[rgb]{0.31,0.60,0.02}{#1}}
\newcommand{\VariableTok}[1]{\textcolor[rgb]{0.00,0.00,0.00}{#1}}
\newcommand{\VerbatimStringTok}[1]{\textcolor[rgb]{0.31,0.60,0.02}{#1}}
\newcommand{\WarningTok}[1]{\textcolor[rgb]{0.56,0.35,0.01}{\textbf{\textit{#1}}}}
\usepackage{graphicx}
\makeatletter
\newsavebox\pandoc@box
\newcommand*\pandocbounded[1]{% scales image to fit in text height/width
  \sbox\pandoc@box{#1}%
  \Gscale@div\@tempa{\textheight}{\dimexpr\ht\pandoc@box+\dp\pandoc@box\relax}%
  \Gscale@div\@tempb{\linewidth}{\wd\pandoc@box}%
  \ifdim\@tempb\p@<\@tempa\p@\let\@tempa\@tempb\fi% select the smaller of both
  \ifdim\@tempa\p@<\p@\scalebox{\@tempa}{\usebox\pandoc@box}%
  \else\usebox{\pandoc@box}%
  \fi%
}
% Set default figure placement to htbp
\def\fps@figure{htbp}
\makeatother
\setlength{\emergencystretch}{3em} % prevent overfull lines
\providecommand{\tightlist}{%
  \setlength{\itemsep}{0pt}\setlength{\parskip}{0pt}}
\setcounter{secnumdepth}{-\maxdimen} % remove section numbering
\usepackage{bookmark}
\IfFileExists{xurl.sty}{\usepackage{xurl}}{} % add URL line breaks if available
\urlstyle{same}
\hypersetup{
  pdftitle={Group 3: Predicting WAR},
  pdfauthor={Gioia Bonanno-Garcia, Ari Crumley, Vincent West},
  hidelinks,
  pdfcreator={LaTeX via pandoc}}

\title{Group 3: Predicting WAR}
\author{Gioia Bonanno-Garcia, Ari Crumley, Vincent West}
\date{2025-12-03}

\begin{document}
\frame{\titlepage}

\begin{frame}
\vspace{0.5cm}

\begin{columns}[T]
\begin{column}{0.48\linewidth}
\begin{itemize}
\tightlist
\item
  Goal: build a model to predict WAR (Wins Above Replacement)
\item
  WAR measures a player's total contribution compared to a
  replacement-level player
\item
  Interpreted as the number of additional wins a player adds to a team
\end{itemize}
\end{column}

\begin{column}{0.48\linewidth}
\begin{figure}
\centering
\pandocbounded{\includegraphics[keepaspectratio]{Pictures/war-rating-chart.jpg}}
\caption{War Ratings Chart}
\end{figure}
\end{column}
\end{columns}
\end{frame}

\begin{frame}{Models used}
\phantomsection\label{models-used}
\vspace{0.5cm}

\begin{columns}[T]
\begin{column}{0.333\linewidth}
\begin{itemize}
\tightlist
\item
  OLS

  \begin{itemize}
  \tightlist
  \item
    Simplest model used
  \end{itemize}
\end{itemize}
\end{column}

\begin{column}{0.333\linewidth}
\begin{itemize}
\tightlist
\item
  LASSO

  \begin{itemize}
  \tightlist
  \item
    Feature selection
  \end{itemize}
\end{itemize}
\end{column}

\begin{column}{0.333\linewidth}
\begin{itemize}
\tightlist
\item
  Boosting

  \begin{itemize}
  \tightlist
  \item
    Tree based method
  \end{itemize}
\end{itemize}
\end{column}
\end{columns}
\end{frame}

\begin{frame}{OLS Models}
\phantomsection\label{ols-models}
\vspace{0.5cm}

\begin{itemize}
\tightlist
\item
  OLS identifies and measures the relationship between a response
  variable and predictor variables.
\item
  Finds a best-fitting line through a set of data points
\item
  Pros: Convenient, accurate regression results for linearly related
  data
\item
  Cons: May be too simplistic for real world examples, assumptions of
  Linear Regression
\end{itemize}
\end{frame}

\begin{frame}{OLS Metrics Plot}
\phantomsection\label{ols-metrics-plot}
\vspace{0.5cm}

\begin{itemize}
\tightlist
\item
\end{itemize}
\end{frame}

\begin{frame}{LASSO Models}
\phantomsection\label{lasso-models}
\vspace{0.5cm}

\begin{itemize}
\tightlist
\item
  LASSO models perform regularization (L1), which shrinks some
  coefficients to exactly zero

  \begin{itemize}
  \tightlist
  \item
    Essentially feature selection
  \end{itemize}
\item
  Pros: Produces a more interpretative model, prevents over fitting
\item
  Cons:
\end{itemize}
\end{frame}

\begin{frame}[fragile]{LASSO metrics plot}
\phantomsection\label{lasso-metrics-plot}
\vspace{0.5cm}

\begin{columns}[T]
\begin{column}{0.4\linewidth}
\begin{itemize}
\tightlist
\item
  Metrics ran on split training data
\item
  Good performance?
\end{itemize}
\end{column}

\begin{column}{0.6\linewidth}
\begin{Shaded}
\begin{Highlighting}[]
\FunctionTok{library}\NormalTok{(ggplot2)}

\NormalTok{lasso\_plot\_pres }\OtherTok{\textless{}{-}} \FunctionTok{data.frame}\NormalTok{(}\AttributeTok{metrics =} \FunctionTok{c}\NormalTok{(}\StringTok{"RMSE"}\NormalTok{, }\StringTok{"MAE"}\NormalTok{), }\AttributeTok{values =} \FunctionTok{c}\NormalTok{(}\FloatTok{0.7329}\NormalTok{, }\FloatTok{0.5594}\NormalTok{))}

\NormalTok{LASSO\_plot }\OtherTok{\textless{}{-}}\NormalTok{ lasso\_plot\_pres }\SpecialCharTok{|\textgreater{}}
  \FunctionTok{ggplot}\NormalTok{(}\FunctionTok{aes}\NormalTok{(}\AttributeTok{x =}\NormalTok{ metrics, }\AttributeTok{y =}\NormalTok{ values)) }\SpecialCharTok{+}
  \FunctionTok{geom\_col}\NormalTok{(}\AttributeTok{fill =} \StringTok{"blue"}\NormalTok{) }\SpecialCharTok{+}
  \FunctionTok{theme\_bw}\NormalTok{() }\SpecialCharTok{+}
  \FunctionTok{geom\_text}\NormalTok{(}\FunctionTok{aes}\NormalTok{(}\AttributeTok{label =} \FunctionTok{round}\NormalTok{(values, }\AttributeTok{digits =} \DecValTok{4}\NormalTok{)), }\AttributeTok{color =}  \StringTok{"white"}\NormalTok{, }\AttributeTok{vjust =} \DecValTok{5}\NormalTok{) }\SpecialCharTok{+}
  \FunctionTok{labs}\NormalTok{( }\AttributeTok{x =} \StringTok{"Metrics"}\NormalTok{, }\AttributeTok{y =} \StringTok{"Values"}\NormalTok{, }\AttributeTok{title =} \StringTok{"Test MAE and RMSE for LASSO Model "}\NormalTok{)}

\NormalTok{LASSO\_plot}
\end{Highlighting}
\end{Shaded}

\pandocbounded{\includegraphics[keepaspectratio]{Presentation_files/figure-beamer/unnamed-chunk-1-1.pdf}}
\end{column}
\end{columns}
\end{frame}

\begin{frame}{Boosting Models}
\phantomsection\label{boosting-models}
\vspace{0.5cm}

\begin{itemize}
\tightlist
\item
  Boosting grows trees sequentially using information from previously
  grown trees

  \begin{itemize}
  \tightlist
  \item
    Each tree fit on a modified version of the original data set
  \end{itemize}
\end{itemize}
\end{frame}

\begin{frame}[fragile]{Boosting metrics plot}
\phantomsection\label{boosting-metrics-plot}
\vspace{0.5cm}

\begin{columns}[T]
\begin{column}{0.4\linewidth}
\begin{itemize}
\tightlist
\item
  Metrics ran on split training data
\end{itemize}
\end{column}

\begin{column}{0.6\linewidth}
\begin{Shaded}
\begin{Highlighting}[]
\FunctionTok{library}\NormalTok{(ggplot2)}

\NormalTok{boost\_plot\_pres }\OtherTok{\textless{}{-}} \FunctionTok{data.frame}\NormalTok{(}\AttributeTok{metrics =} \FunctionTok{c}\NormalTok{(}\StringTok{"RMSE"}\NormalTok{, }\StringTok{"MAE"}\NormalTok{), }\AttributeTok{values =} \FunctionTok{c}\NormalTok{(}\FloatTok{0.7301}\NormalTok{, }\FloatTok{0.537}\NormalTok{))}

\NormalTok{BOOST\_plot }\OtherTok{\textless{}{-}}\NormalTok{ boost\_plot\_pres }\SpecialCharTok{|\textgreater{}}
  \FunctionTok{ggplot}\NormalTok{(}\FunctionTok{aes}\NormalTok{(}\AttributeTok{x =}\NormalTok{ metrics, }\AttributeTok{y =}\NormalTok{ values)) }\SpecialCharTok{+}
  \FunctionTok{geom\_col}\NormalTok{(}\AttributeTok{fill =} \StringTok{"orange"}\NormalTok{) }\SpecialCharTok{+}
  \FunctionTok{theme\_bw}\NormalTok{() }\SpecialCharTok{+}
  \FunctionTok{geom\_text}\NormalTok{(}\FunctionTok{aes}\NormalTok{(}\AttributeTok{label =} \FunctionTok{round}\NormalTok{(values, }\AttributeTok{digits =} \DecValTok{4}\NormalTok{)), }\AttributeTok{color =}  \StringTok{"white"}\NormalTok{, }\AttributeTok{vjust =} \DecValTok{5}\NormalTok{) }\SpecialCharTok{+}
  \FunctionTok{labs}\NormalTok{( }\AttributeTok{x =} \StringTok{"Metrics"}\NormalTok{, }\AttributeTok{y =} \StringTok{"Values"}\NormalTok{, }\AttributeTok{title =} \StringTok{"Test MAE and RMSE for Boosting Model "}\NormalTok{)}

\NormalTok{BOOST\_plot}
\end{Highlighting}
\end{Shaded}

\pandocbounded{\includegraphics[keepaspectratio]{Presentation_files/figure-beamer/unnamed-chunk-2-1.pdf}}

\begin{frame}{Model Comparison: OLS, LASSO, and Boosting}
\phantomsection\label{model-comparison-ols-lasso-and-boosting}
\begin{center}\includegraphics[width=0.33\linewidth]{Pictures/OLS Model Actual vs Predicted} \end{center}

\begin{center}\includegraphics[width=0.33\linewidth]{Pictures/Lasso Actual vs Predicted 2025} \end{center}

\begin{center}\includegraphics[width=0.33\linewidth]{Pictures/Boosting Actual vs Predicted 2025} \end{center}
\end{frame}
\end{column}
\end{columns}
\end{frame}

\end{document}
